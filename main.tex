\documentclass{article}
\usepackage[spanish]{babel}
\usepackage[colorlinks=true, allcolors=blue]{hyperref}
\usepackage[a4paper,top=2cm,bottom=2cm,left=3cm,right=3cm,marginparwidth=1.75cm]{geometry}
\usepackage[backend=biber,style=apa,]{biblatex}
\usepackage{csquotes}
\usepackage{amsmath}
\usepackage{amssymb}
\usepackage{graphicx} 

\addbibresource{ref.bib}

\setlength{\parindent}{0pt}

\newtheorem{example}{Ejemplo}
\newtheorem{definition}{Definición}

% \bibliographystyle{apalike}
% \nocite{*}

\title{Proyecto de Investigación}
\author{Hugo Missael González Cruz}
\date{January 2026}

\begin{document}

\maketitle

\section{Introducción al Análisis de Datos}

La ciencia de datos es la intersección de matemáticas, estadística y programación, con el objetivo
de descubrir patrones en los datos. Según \textcite{Garzon2022}:
\begin{displayquote}
Data science is about solving problems based on
observations of factors (referred to as co-variates, predictors, or just features) 
that may determine a solution. 
\end{displayquote}

El proceso básico se sigue para el análisis de datos es, a grandes rasgos, como sigue:

\begin{enumerate}
    \item Se elige un fenómeno en el mundo real de donde obtenemos los datos en ``crudo''.
    \item Se procesan y limpian los datos, es decir, se transforman de tal forma que 
        sean apropiados para el análisis.
\end{enumerate}

A partir de aquí el proceso no es lineal y depende del objetivo para el que se hace el análisis.
Una opción es hacer un análisis exploratorio de datos. En tal caso, podríamos hallar valores duplicados, faltantes
o mal registrados; de forma que deberíamos regresar al paso anterior, es decir, limpiar de nuevo el conjunto de datos. 
Enseguida, dependiendo del problema que deseamos resolver, podemos usar algún modelo o algoritmo que nos ayude a resolver el 
problema.
Finalmente, se interpretan los resultados y opcionalmente se comunican mediante reportes o visualizaciones.

Podemos agrupar los problemas concernientes al análisis de datos
en tres: clasificación, predicción y \textit{clustering}.
Para describir brevemente cada tipo de problema, llamamos $\Omega$ a la población concerniente 
al problema y una muestra relativamente pequeña (respecto a la población) $D \subset \Omega$. 
Entonces, resolver el problema, consiste
en obtener un modelo $M$ en términos de $D$ que soluciona el problema en los 
términos esperados.

\begin{example}
Considera el problema de clasificar flores Iris en tres categorías: Setosa, Versicolor y Virginica.

En este caso, $\Omega$ es el conjunto de flores Iris, $D$ es el conjunto de flores Iris de las que hemos registrado
datos como la longitud y anchura de los pétalos.
Entonces, una solución sería un algoritmo que, a partir de la longitud y anchura de los pétalos, sea capaz de clasificar una flor Iris en 
la especie correcta.
\end{example}

El anterior es un ejemplo de un problema de \textbf{clasificación}. 
Un problema de este tipo consiste en definir categorías sobre la 
población y, dada una entrada, decidir a cuál de ellas pertenece. 
Esto es, si $\Pi$ es una partición de $\Omega$, dada una entrada $x$, encontrar $C \in \Pi$ tal que $x \in C$.

Una solución a un problema de clasificación es un modelo que coloca a $x$ en la categoría correcta.

Un problema de \textbf{predicción} consiste en, dada una función $f: \Omega \to \mathbb{R}$
que le asigna un valor numérico a alguna característica de $x \in \Omega$ y
cuyo valor es difícil de medir directamente. El problema es hallar el valor de $f(x)$. 
Una solución es un modelo, basado en otras características de $x$,
capaz de predecir el valor de $f(x)$.

Finalmente, un problema de \textit{clustering} o de conglomerados, 
consiste en construir una partición $\Pi$ en $\Omega$, 
dada una métrica $d$ que mide el grado de similitud entre 
elementos de la población. Una solución es un modelo que produce una partición tal que elementos
en un \textit{cluster} son similares.

Por último, 
\begin{quote}
    Data is an objective recording of one or several event(s) in the real world that is 
    accessible at later times for perusal and analysis by at least one person.
\end{quote} \cite{Garzon2022}

\begin{example}
Según la definición anterior, son datos:

\begin{itemize}
    \item Los contenidos de una página web
    \item Mediciones (precisos o no) de algún fenómeno físico
    \item Estadísticas recolectadas por sitios web
\end{itemize}
No son datos:

\begin{itemize}
   \item Los recuerdos de una persona sobre un evento
   \item La ocurrencia de un evento (si no se registra)
   \item Los contenidos de la memoria RAM de una computadora
\end{itemize}

Una forma simple de organizar datos es en una tabla. En un arreglo de este tipo, las columnas
se pueden considerar como vectores de dimensión $n$ ($nD$) que
describen atributos de los datos. Llamamos \textit{records} a las 
columnas. A su vez, los renglones en la tabla 
son un vector de dimensión $p$ ($pD$) conformado por valores de los atributos. Llamamos a un renglón
\textit{feature}. A los valores específicos en una \textit{feature} se llaman \textit{datums} (plural \textit{data}).
De esta manera, una tabla es una vector $nD$ de vectores $pD$.

Aún más, una tabla puede considerarse como una muestra donde las columnas son variables aleatorias. 

Podemos clasificar a las \textit{features} por el tipo de dato que contienen. Tenemos la siguiente clasificación:
\begin{enumerate}
    \item Cualitativa
        \begin{enumerate}
            \item Ordinal
            \item Nominal
        \end{enumerate}
    \item Cuantitativa
        \begin{enumerate}
            \item Discreta
            \item Continua
        \end{enumerate}
\end{enumerate}

Un dato es cualitativo si toma un pequeño número de valores. Es ordinal si es susceptible a ser ordenado y nominal en otro caso.
De forma similar, un dato es Cuantitativo si toma un número grande de valores. Es Discreto si toma una cantidad de valores finito o infinito numerable
y es continua si toma valores sobre un conjunto no numerable.

\end{example}
\section{Análisis Exploratorio de Datos}
Usualmente, se presenta al análisis exploratorio de datos en contraparte al análisis confirmatorio de datos, este último, 
dedicado a las hipótesis y al modelado. Por esta razón, en el análisis exploratorio no hay un modelo explícito.
Según \textcite{Tukey_1977}, un problema básico sobre cualquier conjunto de datos es hacerlo más entendible para la mente.
Para dicho fin es deseable:
\begin{itemize}
    \item Tener una descripción simple
    \item Ser capaces de describir una capa más profunda (aun si no hallamos nada)
\end{itemize}
El análisis exploratorio de datos no puede ser todo el análisis, es un primer paso. Una manera de encontrar pistas que guién
el análisis confirmatorio.
El análisis exploratorio de datos es un método sistemático para hacerse de una visión global 
del conjunto de datos. Para esto, utiliza gráficas, transforma las variables y presenta estadísticas 
que resumen el estado de los datos. Se trata de entender los datos, de hacerse una idea de su forma y ganar intuición.
Sin embargo, no habría gran valor en explorar los datos si descubrimos lo que ya sabíamos. Un buen análisis exploratorio
nos fuerza a ver lo que no esperábamos.

El conjunto de herramientas para hacer análisis exploratorio de datos incluye:

\begin{itemize}
    \item Estadísticas que resumen el estado global del conjunto de datos.
    \item Histogramas, Diagramas de caja, diagramas de tallo y hoja.
    \item Visualizaciones univariadas y multivariadas que ayudan a ver la relación entre las variables.
    \item Técnicas de \textit{clustering} y de reducción de dimensión que 
        ayudan a visualizar conjuntos de datos con una gran cantidad de variables.
\end{itemize}
\section{Reducción de Dimensión}
En general, el objetivo de la Reducción de Dimensión es encontrar una representación en una 
dimensión más baja de los datos y, que a su vez, preserve las propiedades clave para un problema
dado. Dependiendo del enfoque usado, las propiedades que deseamos preservar son distintas.
Nosotros, nos enfocaremos en el enfoque estadístico y en el enfoque geométrico.

Los métodos estadísticos formulan el problema en términos de métricas estadísticas.
En especial, las medidas de dispersión como varianza, covarianza y la correlación entre las 
variables son el principal criterio para evaluar métodos de reducción de dimensión.
Mediante la evaluación de dichas métricas se pueden seleccionar características importantes, 
describir la relación entre las variables, hacer inferencias, clasificar, predecir o agrupar
eventos futuros.

Desde un enfoque geométrico, la idea es definir una noción de distancia apropiada para el problema
y minimizar un función de pérdida que mide las discrepancias entre las distancias cuando se 
reduce la dimensión.
\subsection{PCA}

% \bibliography{ref}
\printbibliography
\end{document}
