\documentclass{article}
\usepackage[spanish]{babel}
\usepackage[colorlinks=true, allcolors=blue]{hyperref}
\usepackage[a4paper,top=2cm,bottom=2cm,left=3cm,right=3cm,marginparwidth=1.75cm]{geometry}
\usepackage[backend=biber,style=apa,]{biblatex}
\usepackage{csquotes}
\usepackage{amsmath}
\usepackage{amssymb}
\usepackage{graphicx} % Required for inserting images


\addbibresource{ref.bib}

\newtheorem{example}{Ejemplo}

% \bibliographystyle{apalike}
% \nocite{*}

\title{Proyecto de Investigación}
\author{Hugo Missael González Cruz}
\date{January 2026}

\begin{document}

\maketitle

\section{Introducción al Análisis de Datos}

La ciencia de datos es la intersección de matemáticas, estadística y programación, con el objetivo
de descubrir patrones en los datos. Según \textcite{Garzon2022}:
\begin{displayquote}
Data science is about solving problems based on
observations of factors (referred to as co-variates, predictors, or just features) 
that may determine a solution. 
\end{displayquote} (p.1).

El proceso básico se sigue para el análisis de datos es, a grandes rasgos, como sigue:

\begin{enumerate}
    \item Se elige un fenómeno en el mundo real de donde obtenemos los datos en ``crudo''.
    \item Se procesan y limpian los datos, es decir, se transforman de tal forma que 
        sean apropiados para el análisis.
\end{enumerate}

A partir de aquí el proceso no es lineal y depende del objetivo para el que se hace el análisis.
Una opción es hacer un análisis exploratorio de datos. En tal caso, podríamos hallar valores duplicados, faltantes
o mal registrados; de forma que deberíamos regresar a limpiar el conjunto de datos. 

Dependiendo de nuestro objetivo ---una vez los datos están limpios--- podemos usar algún algoritmo. 
Finalmente, se interpretan los resultados y opcionalmente se comunican mediante reportes o visualizaciones.

Podemos agrupar los problemas concernientes al análisis de datos
en tres: clasificación, predicción y \textit{clustering}.

Dada la población concerniente a un problema $\Omega$ y una muestra
$D \subset \Omega$, resolver el problema, generalmente, consiste
en obtener un modelo en términos de $D$ que soluciona el problema en los 
términos esperados.

Un problema de \textbf{clasificación} consiste en definir categorías sobre la 
población y, dada una entrada, decidir a cuál de ellas pertenece. Esto es, si $\Pi$ es una partición de $\Omega$, dada una entrada $x$, encontrar $C \in \Pi$ tal que $x \in C$.

Una solución a un problema de clasificación es un modelo que coloca a $x$ en la categoría correcta.

Un problema de \textbf{predicción} consiste en, dada una función $f: \Omega \to \mathbb{R}$
que le asigna un valor numérico a alguna característica de $x \in \Omega$ y
cuyo valor es difícil de medir directamente, hallar el valor de $f(x)$. 
Una solución es un modelo, basado en otras características de $x$,
capaz de predecir el valor de $f(x)$.

Un problema de \textit{clustering} consiste en construir una partición $\Pi$ en $\Omega$, dada
una métrica $d: \Omega \times \Omega \to \mathbb{R}$ que mide el grado de similitud entre 
elementos de la población. Una solución es un modelo que produce una partición tal que elementos
en un \textit{cluster} son similares.\\

\begin{quote}
    Data is an objective recording of one or several event(s) in the real world that is 
    accessible at later times for perusal and analysis by at least one person.
\end{quote} \cite{Garzon2022}

\begin{example}
Según la definición anterior, son datos:

\begin{itemize}
    \item Los contenidos de una página web
    \item Mediciones (precisos o no) de algún fenómeno físico
    \item Estadísticas recolectadas por sitios web
\end{itemize}
No son datos:

\begin{itemize}
   \item Los recuerdos de una persona sobre un evento
   \item La ocurrencia de un evento (si no se registra)
   \item Los contenidos de la memoria RAM de una computadora
\end{itemize}

Una forma simple de organizar datos es en una tabla. En un arreglo de este tipo, las columnas
se pueden considerar como vectores de dimensión $n$ ($nD$) que
describen atributos de los datos. Llamamos \textit{records} a las 
columnas. A su vez, los renglones en la tabla 
son un vector de dimensión $p$ ($pD$) conformado por valores de los atributos. Llamamos a un renglón
\textit{feature}. A los valores específicos en una \textit{feature} se llaman \textit{datums} (plural \textit{data}).
De esta manera, una tabla es una vector $nD$ de vectores $pD$.

Aún más, una tabla puede considerarse como una muestra donde las columnas son variables aleatorias. 

Podemos clasificar a las \textit{features} por el tipo de dato que contienen. Tenemos la siguiente clasificación:
\begin{enumerate}
    \item Cualitativa
        \begin{enumerate}
            \item Ordinal
            \item Nominal
        \end{enumerate}
    \item Cuantitativa
        \begin{enumerate}
            \item Discreta
            \item Continua
        \end{enumerate}
\end{enumerate}

Un dato es cualitativo si toma un pequeño número de valores. Es ordinal si es susceptible a ser ordenado y nominal en otro caso.
De forma similar, un dato es Cuantitativo si toma un número grande de valores. Es Discreto si toma una cantidad de valores finito o infinito numerable
y es continua si toma valores sobre un conjunto no numerable.

\end{example}
\section{Análisis Exploratorio de Datos}
Usualmente, se presenta al análisis exploratorio de datos en contraparte al análisis confirmatorio de datos, este último, 
dedicado a las hipótesis y al modelado. Por esta razón, en el análisis exploratorio no hay un modelo explícito.
Según \textcite{Tukey_1977}, un problema básico sobre cualquier conjunto de datos es hacerlo más entendible para la mente.
Para dicho fin es deseable:
\begin{itemize}
    \item Tener una descripción simple
    \item Ser capaces de describir una capa más profundo (aun si no hallamos nada)
\end{itemize}
El análisis exploratorio de datos es un método sistemático para hacerse de una visión global 
del conjunto de datos. Para esto, utiliza gráficas, transforma las variables y presenta estadísticas 
que resumen el estado de los datos. Se trata de entender los datos, de hacerse una idea de su forma y ganar intuición.
Sin embargo, no habría gran valor en explorar los datos si descubrimos lo que ya sabíamos. Un buen análisis exploratorio
nos fuerza a ver lo que no esperábamos.

\section{Reducción de Dimensión}

\subsection{PCA}

% \bibliography{ref}
\printbibliography
\end{document}
